\documentclass[a4paper,12pt]{article}

\usepackage[utf8]{inputenc}
\usepackage[T2A]{fontenc}
\usepackage[russian]{babel}

\usepackage{geometry}
\geometry{margin=2.5cm}

\usepackage{graphicx}
\usepackage{array}
\usepackage{longtable}
\usepackage{hyperref}
\hypersetup{
    colorlinks=true,
    linkcolor=blue,
    urlcolor=blue
}

\usepackage{enumitem}

\title{Инструкция пользователя системы мониторинга рабочих станций SysMonitor}
\author{Студентка группы ИКБО-30-24 Ефремова Кира}
\date{\today}

\begin{document}

\maketitle

\tableofcontents
\newpage

\section{Введение}

Настоящая инструкция пользователя описывает порядок установки и эксплуатации
системы мониторинга рабочих станций \textbf{SysMonitor}. Система предназначена
для сбора и отображения информации о состоянии удалённых компьютеров, а также
для формирования предупреждений при превышении заданных пороговых значений
ресурсов (CPU, RAM, дисковое пространство).

Документ ориентирован на системных администраторов и технических специалистов,
которые разворачивают сервер SysMonitor и устанавливают агент на рабочие станции
пользователей.

\section{Назначение и состав системы}

\subsection{Назначение системы}

Система SysMonitor решает следующие задачи:

\begin{itemize}[noitemsep]
    \item централизованный сбор метрик нагрузки с рабочих станций;
    \item визуализация данных в виде сводной панели и списка компьютеров;
    \item отображение детализированной информации по каждому компьютеру;
    \item фиксация предупреждений при превышении пороговых значений;
    \item упрощение диагностики проблем производительности.
\end{itemize}

\subsection{Состав системы}

Система состоит из следующих компонентов:

\begin{enumerate}[noitemsep]
    \item \textbf{Сервер SysMonitor} --- веб-приложение на базе Flask с базой данных SQLite, предоставляющее веб-интерфейс администратора и REST API для приёма метрик.
    \item \textbf{Агент мониторинга} (\texttt{agent.py}) --- скрипт на Python, устанавливаемый на рабочую станцию и периодически отправляющий метрики на сервер.
    \item \textbf{Графическое приложение агента} (\texttt{agent\_gui.py}) --- десктоп-приложение с графическим интерфейсом, позволяющее запускать и останавливать мониторинг без использования консоли.
\end{enumerate}

\section{Системные требования}

\subsection{Требования к серверу}

\begin{itemize}[noitemsep]
    \item Операционная система: Windows 10/11 или Linux.
    \item Установленный Python версии 3.10 и выше.
    \item Доступ к сети, общий для сервера и рабочих станций.
\end{itemize}

\subsection{Требования к рабочей станции}

При использовании агента в виде \texttt{.exe}-файла:

\begin{itemize}[noitemsep]
    \item Операционная система: Windows 10/11.
    \item Доступ к серверу SysMonitor по сети (по IP-адресу или имени хоста).
\end{itemize}

При запуске \texttt{agent.py} напрямую:

\begin{itemize}[noitemsep]
    \item Операционная система: Windows или Linux.
    \item Установленный Python, библиотеки \texttt{psutil}, \texttt{shutil} и \texttt{requests}.
\end{itemize}

\section{Установка и запуск серверной части}

\subsection{Клонирование репозитория и подготовка окружения}

На серверной машине необходимо:

\begin{enumerate}
    \item Склонировать репозиторий SysMonitor:
    \begin{verbatim}
git clone https://github.com/kiraefrem0va/SysMonitor.git
cd SysMonitor
    \end{verbatim}

    \item Установить зависимости:
    \begin{verbatim}
pip install -r requirements.txt
    \end{verbatim}
\end{enumerate}

\subsection{Запуск сервера}

Сервер запускается командой:

\begin{verbatim}
python run.py
\end{verbatim}

В типичном случае приложение доступно по адресу:

\begin{verbatim}
http://127.0.0.1:5000
\end{verbatim}

При запуске с параметром \texttt{host='0.0.0.0'} сервер становится доступен
по IP-адресу машины в локальной сети, например:

\begin{verbatim}
http://192.168.1.176:5000
\end{verbatim}

\section{Установка и запуск агента}

\subsection{Запуск консольного агента (agent.py)}

Консольный агент может использоваться для тестирования или в простых сценариях.
Необходимо:

\begin{enumerate}
    \item Скопировать файл \texttt{agent.py} на рабочую станцию.
    \item Открыть файл в текстовом редакторе и указать корректный адрес сервера:
    \begin{verbatim}
server = "http://192.168.1.176:5000"
    \end{verbatim}
    \item Запустить агент:
    \begin{verbatim}
python agent.py
    \end{verbatim}
\end{enumerate}

После этого агент будет периодически собирать метрики и отправлять их на сервер.

\subsection{Запуск графического агента (agent\_gui.py / .exe)}

Для удобства установки на рабочие станции разработано графическое приложение
SysMonitor Agent. Приложение позволяет:

\begin{itemize}[noitemsep]
    \item указать адрес сервера SysMonitor;
    \item задать интервал отправки метрик;
    \item запустить и остановить мониторинг;
    \item наблюдать статус отправки данных.
\end{itemize}

\subsubsection*{Работа с .exe-файлом}

Для использования на машинах без установленного Python возможно создание
исполняемого файла (\texttt{agent\_gui.exe}) с помощью PyInstaller. В этом случае
пользователю необходимо лишь:

\begin{enumerate}
    \item Скопировать файл \texttt{agent\_gui.exe} на рабочую станцию.
    \item Запустить \texttt{agent\_gui.exe} двойным щелчком.
    \item Ввести адрес сервера и нажать кнопку <<Запустить мониторинг>>.
\end{enumerate}

\begin{figure}[h]
    \centering
    \includegraphics[width=0.9\textwidth]{images/agentgui.png}
    \caption{GUI агента SysMonitor}
\end{figure}

\section{Работа с веб-интерфейсом}

\subsection{Авторизация}

При переходе на адрес сервера SysMonitor в браузере пользователь видит форму
авторизации. В базовой конфигурации для тестовых целей используется простой
логин/пароль, заданный в серверном приложении (\texttt{admin/admin}).

\begin{figure}[h]
    \centering
    \includegraphics[width=0.7\textwidth]{images/login.png}
    \caption{Окно авторизации в системе SysMonitor}
\end{figure}

После успешной авторизации пользователь перенаправляется на главную панель.

\subsection{Главная панель}

Главная панель отображает сводную информацию:

\begin{itemize}[noitemsep]
    \item количество подключённых компьютеров;
    \item количество активных проблем (предупреждений);
    \item среднюю загрузку CPU по всем рабочим станциям;
    \item таблицу текущих предупреждений.
\end{itemize}

\begin{figure}[h]
    \centering
    \includegraphics[width=0.9\textwidth]{images/dashboard.png}
    \caption{Главная панель администратора SysMonitor}
\end{figure}

\subsection{Список компьютеров}

Раздел <<Компьютеры>> отображает карточки всех машин, от которых поступают
метрики. Для каждой машины показываются:

\begin{itemize}[noitemsep]
    \item имя хоста;
    \item текущая загрузка CPU;
    \item использование оперативной памяти;
    \item заполненность диска;
    \item время последнего обновления.
\end{itemize}

В верхней части страницы расположен поисковый фильтр по имени компьютера.

\begin{figure}[h]
    \centering
    \includegraphics[width=0.9\textwidth]{images/computers.png}
    \caption{Список компьютеров в системе SysMonitor}
\end{figure}

\subsection{Детальная информация о компьютере}

При переходе по ссылке <<Открыть>> или клику по имени хоста пользователь попадает
на детальную страницу компьютера. На ней отображаются:

\begin{itemize}[noitemsep]
    \item последняя полученная метрика;
    \item история последних измерений (например, 20 последних записей);
    \item числовые значения CPU, RAM, диска и количества процессов.
\end{itemize}

\begin{figure}[h]
    \centering
    \includegraphics[width=0.9\textwidth]{images/computerdetail.png}
    \caption{Детальная информация об удалённом компьютере в системе SysMonitor}
\end{figure}

\section{Настройки оповещений}

\subsection{Пороговые значения}

В разделе <<Настройки оповещений>> администратор может задать пороговые значения,
при превышении которых формируются предупреждения.

\subsection{Применение порогов в системе}

Пороговые значения влияют на формирование предупреждений в главной панели.
Пример их использования приведён в таблице~\ref{tab:thresholds}.

\begin{table}[h]
    \centering
    \caption{Пример использования пороговых значений в SysMonitor}
    \label{tab:thresholds}
    \begin{tabular}{|>{\raggedright\arraybackslash}p{4cm}|>{\centering\arraybackslash}p{3cm}|>{\raggedright\arraybackslash}p{6cm}|}
        \hline
        \textbf{Метрика} & \textbf{Порог} & \textbf{Описание предупреждения} \\
        \hline
        Высокая загрузка CPU & 85\% & \\
        \hline
        Высокая загрузка RAM & 80\% & \\
        \hline
        Диск: & 90\% & заполнен \\
        \hline
    \end{tabular}
\end{table}

\begin{figure}[h]
    \centering
    \includegraphics[width=0.9\textwidth]{images/alerts.png}
    \caption{Настройка пороговых значений}
\end{figure}

\section{Заключение}

В данной инструкции пользователя описаны назначение и состав системы SysMonitor,
процедура установки серверной части и агента, а также порядок работы с
пользовательским интерфейсом. Система может быть использована в учебных целях
для демонстрации принципов мониторинга рабочих станций, а также служить
основой для дальнейшего расширения функциональности и интеграции с другими
сервисами.

\section*{Список использованных источников}

\begin{thebibliography}{9}
\bibitem{flask}
Документация проекта Flask. URL:
\url{https://flask.palletsprojects.com/}

\bibitem{psutil}
Документация библиотеки psutil. URL:
\url{https://psutil.readthedocs.io/}

\bibitem{pytest}
Документация по pytest. URL:
\url{https://docs.pytest.org/}
\end{thebibliography}

\end{document}
